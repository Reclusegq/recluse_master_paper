\chapter{基于智能手机的行为识别研究概述}
\par 人体行为识别就是通过感知设备获取关于人体行为的数据,然后通过信号处理和模式识别等技术手段对用户当前行为作出识别判断,其整体流程如图所示。行为识别的概念在上世纪八十年代就已经被提出,而在研究初期它仅仅局限在计算机视觉领域。研究者通过使用摄像头等作为感知设备获取关于人体的图像序列或视频片段,通过图像处理等技术手段提取特征并识别用户行为。随着微电子和集成电路等技术的发展,开始出现一些小型化,功能强大的传感器,一些研究者开始关注基于可穿戴传感器的行为识别技术。基于可穿戴传感器的行为识别主要使用加速度传感器采集人体的行为数据,然后提取时域和频域等特征进而识别用户行为。
\par 近些年,随着智能手机的不断发展,其内部集成了越来越多的传感器,不仅包括在基于可穿戴传感器的行为识别中使用最多的加速度传感器,还包含有磁场传感器,陀螺仪等运动类型传感器,此外还有光照传感器,压力传感器,GPS等环境信息感知设备,可以获取充足的关于用户行为和环境信息。另一方面,智能手机的存储和计算能力也在不断提高与发展,其功能不断拓展与增强,这些就为基于智能手机进行行为识别提供了良好的硬件基础。伴随着智能手机的普及与应用,研究者也开始将重点关注到基于智能手机的行为识别技术。
\par 本章首先介绍智能手机内置的主要传感器以及其他感知设备,然后介绍在行为识别过程中必要的数据预处理步骤以及常用的特征提取和特征选择方法,最后详细介绍部分可以用于智能手机行为识别的分类模型和算法。

\section{数据采集}
\subsection{智能手机内置传感器简介}
\par 对于智能手机的内置传感器的介绍,本文以Android操作系统为例。Android操作系统是一种基于Linux的自由及开放源码的操作系统,主要用于移动设备,有Google公司和开放手机联盟领导开发。之所以选择Android系统为例介绍,一方面是因为Android是一款开源且成熟的移动端操作系统,另一方面它也也是目前市场占有量最大的手机操作系统,因此更具有代表性。对于传感器的集成以及为应用提供的传感器数据服务,其他操作系统也都十分类似,本节只是不是一般性地以Android系统为例介绍智能手机中的内置传感器。
\par Android系统平台支持的传感器类型很广泛,可以从功能和实现方式两个维度对其分类。一方面按照功能的不同,传感器可以分为运动类型传感器,环境传感器和方位传感器三类。运动类型传感器主要用于测量三轴向的加速度和角速度等,通常用于监测设备的移动,比如倾斜、震动、旋转或者摇摆等,主要包括加速度传感器、重力传感器、陀螺仪和旋转矢量传感器等。这一类型传感器在大部分智能手机内部都有集成,而且运动类型的数据也可以较好地用于人体行为识别。环境传感器则适用于测量各类环境信息参数,例如外界环境的温度、大气压强、光照强度和空气湿度等,主要包括温度传感器,气压传感器,温度传感器等。这一类型传感器在部分智能手机内有集成,可以检测人体周围环境信息,辅助人体行为识别,但是没有普适性。方位传感器主要用于测量设备的方向,主要包括磁场传感器和方向传感器,此外一般智能手机都集成了近距离创拿起,可以用于检测设备表面与物体的距离。
\par 另一方面按照实现方式的不同,传感器可以分为基于硬件和基于软件两类。基于硬件的传感器是集成在移动终端设备的物理实体。它们获得数据是通过直接测量特定的环境信息,比如加速度传感器,陀螺仪等。软件传感器是通过模拟硬件传感器,而并非真实的物理设备,它们是通过整合一个或多个硬件传感器信息相应数据,并通过一些融合方法计算获得相应模拟传感器的数据,因此也被称为虚拟传感器或者人工传感器,例如重力传感器,方向传感器等。Android平台所支持的传感器总结如下表。

%传感器分类表
\begin{table}[htbp]
\centering
\caption{Android系统平台下内置传感器分类表}
\begin{tabular}{|c|c|c|}
    \hline
    分类 & 基于硬件的传感器 & 基于软件的传感器 \\
    \hline
    运动传感器 & 加速度传感器,陀螺仪 & 线性加速度,旋转矢量传感器\\
    \hline
    环境传感器 & 光照传感器,气压传感器等 &  \\
    \hline
    方位传感器 & 磁场传感器,近距离传感器 & 方向传感器,重力传感器 \\
    \hline
\end{tabular}
%\note{这里是表的注释}
\end{table}

\par 下面本文详细介绍智能手机中常用于行为识别的传感器,包括加速度传感器,陀螺仪和磁场传感器。
\begin{itemize}
	\item 加速度传感器
\end{itemize}
\par 在Android系统平台下,存在加速度和线性加速度两类测量设备加速度,前者是基于硬件的传感器,又开发商生产直接集成在智能手机内部,产生原始的加速度数据,该数据包含重力加速度的影响。而后者则是基于软件的传感器,它是有加速度数据和其他类型传感器的数据融合后计算而得到的加速度信息,排除了重力加速度的影响。二者都会提供$x, y, z$三轴方向的加速度数据,其中加速度数据不仅提供加速度的大小(以$m/s^2$为单位),还提供了加速度的方向。对于每个轴向,加速度存在正值和负值分别表示两个相反方向的加速度。三个轴向的定义如图所示。

\par 运动和方向传感器都是采集的矢量数据,数据的方向通常可以定义在直角坐标系中。Android系统平台定义了两个直角坐标系用以表征设备方向的相对变化。两个坐标系包括全局坐标系 $x_E, y_E, z_E$ 以及设备坐标系$x, y, z$,其相对关系如图。图中表示了智能手机略微倾斜地放置在地球赤道上方时方向相对关系。在全局坐标系中$y_E$指向磁场北极,即接近正北方, $x_E$平行于地球表面,与$y_E$垂直, $z_E$指向正上方,即远离地心的方向。在设备坐标系中, $x$轴为水平方向,向右为正,$x$轴为垂直方向,向上为正,$x$轴为垂直于屏幕方向,屏幕正前方为正。几乎所有的三轴向的传感器都符合该坐标系包括下面会介绍的陀螺仪和磁场传感器,其所采集的数据均为全局坐标系中相应物理量在设备坐标系中三轴向的分解。
\par 加速度数据可以很好地表征不同行为的特点,加速度传感器在行为识别中也是应用最为广泛。基于可穿戴传感器的行为识别研究中,基于所有文献都是基于的加速度数据,而在智能手机中,加速度传感器集成也十分广泛,因此本文中它被选择为用于研究行为识别的传感器之一。

\begin{itemize}
	\item 陀螺仪
\end{itemize}
\par 陀螺仪是在设备发生旋转时,通过测量科氏(Coriolis)力测量设备在三个轴向上的角速度或旋转速度。所谓科氏力是指使得自由旋转物体从旋转参考系中看起来发生偏移的力。陀螺仪只能测量角速度,而不能直接测量旋转角度。但是可以通过陀螺仪的测量值在时间上的积分计算设备的旋转角度。陀螺仪的输出时绕设备坐标系的$x, y, z$三个轴向的旋转角速度值,单位为$rad/s$如果坐标轴指向你自己,逆时针旋转则为正值,反之为负值。
\par 陀螺仪可以检测设备的旋转角速度,而在行为识别中,大部分的动态行为都是周期性行为,设备也会发生周期性旋转,所以通过处理陀螺仪数据可以很好地提取表征动态行为的特征,很好地辅助与行为识别。

\begin{itemize}
	\item 磁场传感器
\end{itemize}
\par 不同生产厂商的磁场传感器实现方式会有所不同。它们的实现方式主要可以分为基于霍尔效应、磁阻材料和洛伦兹力三种。其中霍尔效应传感器占据了最大的市场份额,在电流通过导线时,存在的磁场会使得导线两端的电子密度不同,从而形成正比于磁感应强度的电势差,通过电势差即可以计算出磁感应强度。磁场传感器也会以$x, y, z$三个轴向输出磁感应强度,因为集成的传感器是在每个方向上都有一个传感器。输出数据以微特斯拉为单位。
\par 磁场传感器的输出可以很好地判断设备方向的变化。在行为识别中,当行为发生转变时,磁场传感器的输出也会起到很好的辅助作用。

\subsection{智能手机的数据采集模块}
\par 基于可穿戴传感器的行为识别中数据采集都是传感器将采集数据通过近距离通信的方式传送至中心节点,而基于智能手机的行为识别,其传感器和数据处理中心都是集成在手机内部,因此我们需要智能手机操作系统提供的编程接口,提取手机内置传感器的采集数据。Android系统提供了一系列的应用编程接口(API, Application Program Interface),方便获取传感器的数据。有关传感器的操作都有SensorManager负责同一管理。

\begin{enumerate}[a)]
	\item 基本API
\end{enumerate}
\par SensorManager时Android为便于应用访问传感器数据提供的系统服务。通过SensorManager一方面可以获取手机内置的传感器对象,用Sensor类表示设备种的传感器。另一方面通过SensorManager可以允许应用注册或注销传感器相关事件,并在注册成功后可以接受传感器数据。传感器事件可以通过注册的监听器SensorEventLitener监听。注册传感器事件以获取传感器数据,需要提供对于的Sensor对象,SensorEventListener对象和数据采集的采样时间间隔等。注册成功以后即可以通过定义的SensorEventListener中的回调方法获取传感器数据。

\begin{enumerate}[b)]
	\item 传感器采样率
\end{enumerate}
\par 在注册监听器时,通过指定监听器的采样间隔决定数据的采样率。在Android系统中,提供有四个常量表示采样间隔:SENSOR\_DELAY\_FASTEST ($0ms$,硬件传感器决定采样率),SENSOR\_DELAY\_GAME ($20ms, 50Hz$),SENSOR\_DELAY\_UI ($67ms, 15Hz$), SENSOR\_DELAY\_NORMAL ($200ms, 5Hz$)。同样也可以通过指定其他的事件间隔,即指定采样率注册事件监听器以获取相应传感器数据。

\begin{enumerate}[c)]
	\item 传感器数据
\end{enumerate}
\par Android系统中传感器的数据是以一个SensorEvent数据结构表示,并由传感器传递至应用定义在监听器SensorEventListener中的回调方法中,从而将传感器数据传回到应用中。SensorEvent包含以下属性用以保存数据:
\begin{itemize}
	\item SensorEvent.accuracy:表示传感器当前采集数据的精度
	\item SensorEvent.senor:表示采集该数据的传感器
	\item SensorEvent.timestamp:表示采集到该数据的时间
	\item SensorEvent.values:表示传感器的数据,以数组的形式保存三轴向的数据
\end{itemize}