\chapter{绪论}

%\section{模板简介}

%测试脚注\footnote{分别编号}。

\section{研究背景与研究意义}
\par 近年来,随着硬件设备和软件技术的不断进步与发展,各类小型化,具有感知、通信和计算能力的设备开始普及。伴随而来的是围绕着人的需求而展开的各类服务技术逐渐多样化和智能化。在各类服务应用中,一项重要的技术支撑就是人体行为感知和识别技术,即通过人体周围的感知设备,获取用户行为相关的数据,借助于现有信号处理方法和模式识别技术,对用户当前行为做出识别和判断,从而提供更加智能化的服务。基于行为识别的应用十分广泛,在医疗方面,行为识别可以用于检测老年人或者患者的异常行为举动,比如跌倒,以便于及时发现和救助。在人机交互方面,系统根据用户行为判读其需求及意图,及时进行系统控制调整,减少用户主动参与性。在智能家居场景中,家庭管理系统根据用户当前行为判断用户需求,然后控制家居设备做出调整,使得服务更加智能化。另外,在体育和娱乐方面行为识别也有着广泛的应用,比如随身运动监测、体感游戏等。
\par 行为识别技术从上世纪80年代开始发展,到目前为止,根据感知设备的不同,主要分为两大类,基于计算机视觉的行为识别和基于可穿戴传感器网络的行为识别。前者主要利用摄像头作为感知手段,通过处理拍摄得到的图像序列识别其中人的行为。图像序列易于被人所理解而且计算机视觉相关技术发展也日臻成熟与完善,但是利用摄像头进行行为识别却有着其固有的局限性。首先是个人隐私问题,通过摄像头拍摄的图像过多地暴露了用户的个人信息,尤其不适用于智能家居的应用场景,其次是摄像头的部署只能局限在部分区域,不利于长期行为监测与识别。相比较而言,基于可穿戴传感器网络则具有随时随地感知与识别并且隐私侵入性小等优点。但是,随之而来的问题是用户必须佩带多个特定位置的传感器,用户体验较差,不利于行为识别技术的应用与普及。
\par 鉴于上述技术存在的问题,近几年有人提出基于智能手机的行为识别方法。随着嵌入式技术的发展和硬件工艺水平的不断提高,传感器模块体积越来越小。各个硬件厂商为满足用户不断增加的需求,将越来越多的传感器设备集成到智能手机中。智能手机逐渐成为集感知设备、数据处理设备以及无线通讯设备为一体的智能化综合平台,拥有着强大的计算能力和数据通讯能力。因此,基于智能手机就可以实现从底层的数据采集到中间层的数据处理,再到上层的应用开发和服务提供。另一方面,随着智能手机的普及以及其与现代人的密切关系,智能手机内置传感器可以更加方便地采集人体行为相关数据,且没有额外的传感器负担和侵入性,而且无需考虑传感器节点的能耗问题以及传感器节点与中心节点的通信问题。因此,基于智能手机的行为识别具有使用方便和侵入性小等优点,便于长期监测,有利于行为识别技术的应用和普及。
\par 虽然基于智能手机的行为识别相比较于可穿戴传感器网络,使用方便灵活且侵入性小,但是随之带来了一些其他的问题和挑战。首先,不同与可穿戴传感器,智能手机相对于人体的方向并非固定的,因此采集到的各轴向的传感器数据会随着方向的变化而变化。其次,智能手机智能采集人体一个位置的数据,而可穿戴传感器则可以通过部署在人体多个位置的传感器采集更加有效的数据。最后,位置敏感性是其中最大的问题。当手机处于人体不同位置时,即使在相同行为下,传感器所采集数据也会具有很大不同。这些都将会在行为识别过程中造成较大的影响。
\par 因此,本文基于智能手机的行为识别的研究,重点解决方向变化和位置变化的问题具有重要的意义。
\section{国内外研究现状与发展趋势}
\par 最近几年,有关行为识别的研究有很多,但是因为方向和位置不确定性等因素,基于智能手机行为识别相关的研究并不多。智能手机内置传感器有多种,主要分成运动传感器和环境传感器两类。而在行为识别中,大部分工作主要集中在运动传感器的使用,尤其是加速度传感器。Muhammad Shoaid et al在[2]中通过对比实验研究了除加速度传感器以外,陀螺仪和磁力计对行为识别的影响,实验分别对不同位置和不同行为分别做了对比实验,结果表明它们对不同位置的不同行为所起到的作用会有所不同,且区别较大,比如陀螺仪在区分上下楼和行走的过程中则具备较好效果,对其他行为区分度则不大。而使用上述三类手机内置传感器,最大的问题是手机的方向和位置不确定性。为了避开智能手机的方向和位置变化问题,在一些文章中[3] [4]中假定手机的方向和位置固定不变,并且规定手机相对于人体的方向,重点研究三轴方向的某几个方向的特征,在实验中将手机绑定在裤子口袋位置,不过这种假设过于严格与实际相差较大,不利于行为识别应用的推广。
\par 对于方向的变化问题,一方面可以使用方向独立的数据提取特征,M.B.Rasheed 等人在[5]中使用传感器的三轴方向数据的信号模值向量最后的数据用于特征提取,从而避免方向变化带来的影响,另一方面利用旋转矩阵的方法作用于数据向量,进而减少方向变化带来的干扰。Surapa Thiemjarus 等人在[6][7]中提出一种基于投影的方法,各轴向数据间的相关系数矩阵进行特征值分解,进而计算旋转矩阵对数据进行旋转变换,减弱因方向变化对数据产生的干扰。虽然基于旋转矩阵可以一定程度上解决智能手机方向变化的问题,但是计算得到的旋转矩阵是手机坐标相对于地球坐标的角度,旋转以后的数据无法反映出用户身体相对于地球坐标系的变化,在一些行为的识别过程中丢失了方向信息。
\par 对于位置的变化问题,一方面是使用位置独立的特征或分类方法,Changhai Wang等人在[8]中提出频域的特征受到位置变化的影响较小,使用频域特征时,在相同行为下,手机被放置在不同位置行为识别正确率较高。另一方面则是针对每个位置训练不同的分类模型,在识别阶段首先识别其位置信息,然后使用不同的分类模型对行为进行识别分类。在解决位置变化的问题上,前一个方法中寻找对于位置不敏感的特征比较受限,而后者在研究中[14]比较表明位置独立的模型分类的总体效果反而不如所有位置的统一分类模型。
\par 除使用运动传感器以外,部分研究尝试使用智能手机内置的环境传感器,通过检测用户周围的情景信息辅助行为检测。A.E.Halabi 等人在[10]中研究通过结合使用手机内置压力传感器检测周围大气压力的变化用于区分上楼和下楼两种行为。R.D.Das 等人在[11]中则利用GPS信息辅助行为识别的最终决策。
\par 以上的研究都是在手机移动端如何实现行为识别以及如何提高识别准确率,但是在移动端这个特殊的生态环境下,能量受到一定的限制,因此如果降低在识别过程中的能耗也是这项研究中很重要的一方面。
\par 关于降低能耗的研究主要集中在寻找一种在识别准确率和能耗之间权衡策略。V.Q.Viet 等人在[12]以及Zhixian Yan 等人在[13]中均以采样率和特征集作为自适应调整对象,针对不同行为的特点选择最佳的采样率和特征集的组合。两篇文章中均使用窗口平滑决策的策略,以及设置置信度阈值判断是否发生行为转变的方法,即利用行为相对持续不变的特点,包括调整采样率等措施,从而在保证一定的识别准确率的基础上有效降低能耗。使用智能手机进行行为识别主要的能耗都在于数据采集过程中,上述方法中虽然采取了选择合适采样率,但是并没有考虑到行为的连续性而采取减少数据采集时间的方法,从而进一步减少在整个过程的能量消耗。
\subsection{基于智能手机的行为识别}
\subsection{关于降低行为识别过程中能耗的方法}
\section{研究内容与贡献}
\section{本文结构与安排}

\begin{figure}[ht]
\centering
\includegraphics[width=10cm]{ustc_logo_fig}
\caption{测试图片}
\note{the solid lines represent the time histogram of the spontaneous activities of an old monkey cell(gray) and a young monkey cell (black). The bin-width is 1}
\end{figure}


\subsection{模板介绍2}

\begin{table}[ht]
\centering
\caption{测试表格}
\begin{tabular}{cc}
A   &   B   \\
C   &   D   \\
\end{tabular}
\note{测试表注}
\end{table}



\section{系统要求}



\section{问题反馈}
测试脚注\footnote{脚注2}
