\begin{abstract}
\par 人体行为识别是通过感知设备获取与人体行为相关的数据,并利用信号处理方法以及模式识别技术对人体当前行为作出识别判定,它广泛应用于医疗健康、人机交互、智能家居、体育娱乐等诸多方面。传统的行为识别技术主要分为基于计算机视觉技术和基于可穿戴传感器技术两类。基于计算机视觉技术的行为识别是通过图像或视频识别人体的行为,但是它却存在着区域局限性和侵犯用户隐私等问题。基于可穿戴传感器的行为识别则是通过人体佩戴加速度等传感器获取人体运动信息,进而判断人体行为,但是着需要用户佩戴一定数量传感器,存在用户侵入性、传感器节点的通信以及节点寿命等问题。因此,有研究者开始考虑使用智能手机进行行为识别。智能手机内置多种类型传感器且容易采集人体运动数据,此外不存在数据通信以及节点寿命等需要考虑的问题,因此基于智能手机的行为识别使用更加灵活方便。然而,由于智能手机的灵活性,它在人体上所处方向以及所处的位置都会不断发生变化,这对与采集运动数据进行行为识别造成很大的影响。此外,虽然智能手机充电十分方便,但是由于移动端能量毕竟受限,数据采集与识别能耗也是迫切需要关注的问题。为了在智能手机终端实现一个使用方便,灵活,低能耗的行为识别应用,本文考虑针对基于智能手机的行为识别技术进行了研究和设计:
\begin{enumerate}[(1)]
	\item 基于智能手机的两层多策略行为识别框架。该框架旨在应对基于智能手机的行为识别技术中存在的手机方向和位置变化问题。框架流程中首先对传感器数据做分段预处理并提取特征,本文采样时域、频域和自相关函数三种类型的行为特征,特别是本文提出使用自相关函数特征降低方向变化的干扰。然后在分类过程中采样两层多策略的方法,即将行为分组以后根据不同特点采样不同策略进一步分类。对于静态行为通过识别整个过程中过渡状态行为进行判断,而对于动态行为,引入位置分类器获取手机位置信息并利用特定位置数据训练的分类器对行为做进一步分类。最后,分类实验结果表明该框架可以提高识别准确率,特别是对于慢速的动态行为,相比于没有采用位置分类器获取位置信息的方法,准确率有较大幅度的提升。
	\item 降低识别能耗的最佳策略动态调整方案。降低识别能耗的基本方法就是通过调节采样率等变量权衡能耗与准确率的关系,最终选择一个最佳的策略进行行为识别。为此,本文首先对能耗和准确率建立了数学模型,研究二者与采样率、采样时间以及特征选择范围之间的函数关系,并以电量为权衡系数结合二者建立目标函数,从而将最佳策略选择转化为求解关于目标函数的最优化问题。然后通过实验测量的方法求解数学模型中的参数,为每一种行为求解出一种最佳的识别策略。同时,为了在识别过程中合理调整策略,本文提出一种结合行为转换和识别结果的转换判定方法,动态地调整最佳策略进行行为识别。最后,对比实验结果分析分析表明本文所提出的最佳策略动态调整方法对于降低识别能耗十分有效。
	\item 行为识别手机应用设计。最后我们在智能手机端编写了行为识别应用,实现了本文所提出的方法。该应用按照其功能可以分为控制层、算法实现部分、模型层和视图层四部分,分别负责协调数据通信,实现分类算法,存储数据和展示结果的功能。
\end{enumerate}

\keywords{人体行为识别\zhspace 智能手机\zhspace 传感器\zhspace 自相关函数特征\zhspace
位置分类器\zhspace 低能耗\zhspace 策略调整\zhspace 手机应用}
\end{abstract}

\begin{enabstract}
This is USTC thesis template for bachelor, master and doctor user's guide.
The template is created by ywg@USTC and a derivative of USTC Bachelor and
Master-PhD templates. Besides that the usage of the template, a brief guideline
for writing thesis is also provided.

\enkeywords{University of Science and Technology of China (USTC), Thesis, Universal \LaTeX{} Template, Bachelor, Master, PhD}
\end{enabstract}
