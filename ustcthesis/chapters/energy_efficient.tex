\chapter{降低识别能耗的策略动态调整研究}
\par 在移动终端这个特殊的系统环境下,其中最大的限制就是能量有限。而在基于智能手机的行为识别整个过程中无论是内置传感器采集数据还是计算特征并对当前行为进行判决分类都会消耗较大的能量。因此我们有必要要研究在行为识别过程中降低手机能耗的方法,其最终目的就是通过权衡识别准确率和识别能耗,在保证一定准确率的前提下,尽可能降低在识别过程中智能手机的能耗。
\par 使用智能手机进行行为识别过程中,传感器采集运动数据,处理器计算特征向量以及运行分类算法对行为做判断识别都会消耗智能手机较多的能量。而在这一过程中,能耗的大小与传感器的采样率,采样时间,以及特征向量和分类算法的计算复杂度都有着直接的关系。另一方面,行为识别的识别准确率则取决于采集数据的充分性,特征集和分类算法的选择,所以准确率同样与采样率,采样时间,特征向量等的选择有直接关系。与此同时,对于不同的行为对数据量的要求,对特征集合的选择等都存在很大的不同,比如静止行为状态下,运动信息很少,对数据量的要求不高,而且也不存在频域特性,因此不需要在这段时间内以较高的采样率采集数据并计算频域等复杂特征,只需要运行较低的采样率监测过渡态行为是否发生即可。又比如在跑步状态下,其频域特征十分明显,且较容易识别,此时传感器只需要在窗口时间内的一部分时间采集数据就可以识别跑步状态,其他时间传感器进入休眠可以有效降低能耗。因此可以针对不同行为通过调节上述的一些变量,在较少影响准确率的情况下尽可能降低识别能耗。
\par 为研究降低识别能耗的策略调整方法,本章首先建立能耗和准确率的数学模型,研究二者与一些变量的函数关系,包括采样率,采样时间,特征集的选择等。与此同时,为了权衡识别准确率和识别能耗的关系,本文使用手机当前的能耗作为权衡系数,建立目标函数,为每一类行为求解最佳策略只需要求解相应目标函数的最优值即可,即转换为最优化问题。然后通过实验测量数据求解数学模型参数,进而求解每类行为的目标函数的最优解,从而获得每一种行为的最佳策略。最后提出一种结合行为识别结果和马尔科夫模型的行为转换矩阵的策略动态调整方法。

\section{数学模型}
\par 对于识别准确率和识别能耗的影响因素,本文考虑采样率,采样时间以及